\section{Results}
\subsection{Baseline}
For our first test we need to establish a baseline to compare all other results against.  We use the even-3500 dataset which included  description-unigrams for this baseline.  A support vector machine and maximum entropy algorithm were trained on this dataset.  The support vector machine attained 49.33\% for its testing accuracy. The testing accuracy for maximum entropy was 46.12\%.  Because there are 7 categories, random chance would put the accuracy at 14.28\%.  This means that, as a baseline, the algorithms are already performming above random chance.

\subsection{Dimensionality Reduction}
Due to the large number of features, we first attempted dimensionality reduction on the dataset as a means to increase the accuracy of the algorithm.  We attempt two dimensionlaity reductions, principle component analysis and laplacian eigenmap. Both processes were applied to the even-3500 dataset of descrption-unigrams.  We found that the results tended to be ambiguous, in that the testing accuracy of maximum entropy tended to increase by approximately 7\% while the support vector machine had its testing accuracy decreased by 7\%.  However, except for these adjustments, the results did not change significantly.
\\

\begin{figure}[!h]
\begin{center}
\caption{Maximum Entropy with Dimensionality Reduction Testing Accuracy}
\includegraphics[width=0.7\textwidth]{Maximum_Entropy_Dimensionality_Reduction.png}
\end{center}
\end{figure}

\begin{figure}[!h]
\begin{center}
\caption{SVM with Dimensionality Reduction Testing Accuracy}
\includegraphics[width=0.7\textwidth]{SVM_Dimensionality_Reduction.png}
\end{center}
\end{figure}

Overall, it is interesting to note that there tended to be early convergence in the number of features.  Specifically, maximum entropy converged by around the 10 dimensionality mark, and the difference between 100 and 1000 for the support vector machine was significant for PCA, however, it was not significant for the Laplacian eigenmapA.  This seems to suggest that in natural language tasks, dimensionality reduction quickly converges to find the important features that should be used in data processing.  Though this is an interesting result, to have an accurate comaprison against our baseline, further results do not include dimensionality reduction and focus instead on dataset size and feature selection.

\subsection{Dataset Size}
For the third set of experiments we wanted to determine to what extent more data would begin to alter the accuracy the algorithms.  We use the the jagged-20000 and jagged-40000 datasets which each included description-unigrams.

\begin{figure}[!h]
\begin{center}
\caption{Testing Accuracy with \emph{description-unigrams} features}
\begin{tabular}{| r | c | c |}
\hline
Dataset & SVM & Maximum Entropy \\ \hline
\emph{even-3500} & 49.33\% & 46.12\% \\ \hline
\emph{jagged-20000} & 66.8667\% & 69.21\% \\ \hline
\emph{jagged-40000} & 69.53\% & 69.75\% \\ \hline
\end{tabular}
\end{center}
\end{figure}

The size of the dataset noticeably improved the results from 3500 to 20000.  However, after this point it appears that the algorithm converged.  This is seen from the change from 20000 to 40000 samples.  The increase was nearly insignificant for the maximum entropy algorithm.  To summarize, data is able to improve the testing accuracy up to a point, then gains are very small in comparison to early gains.  To further improve the results, either a different classification algorithm would need to be used, or a different set of features would need to be included.  To this end, the following set of experiments will attempt new ways of calculating the feature-vectors for each patent within the 40000-jagged dataset used in this experiment.

\subsection{Abstract Bigrams}
For the fourth experiment, the jagged-40000 dataset was used with abstract-bigrams used as the features.  The results indicate that the support vector machine had an accuracy of 72.12\% and the maximum entropy algorithm had an accuracy of 72.57\%.  Overall, this is a slight improvement from experiment 3 of 3\% for the support vector machine and an increase of 3\% for the maximum entropy classifier.

\begin{figure}[!h]
\begin{center}
\caption{SVM with Dimensionality Reduction Testing Accuracy}
\includegraphics[width=0.7\textwidth]{Unigrams_vs_Bigrams.png}
\end{center}
\end{figure}

The small gains from this experiment are strictly due to the use of abstract-bigram features.  This is known from the fact that only item which changed between the 40000-jagged dataset in the previous set of experiments was the move from description-unigrams to abstract-bigrams.  These results seem to suggest that it is still possible that further gains can be made by creating better features. Therefore, it is still of interest to attempt to create more intelligent features while using the largest possible dataset that we have.  The next experiment will attempt this experiment. 

\subsection{tf-idf}
We then used tf-idf on the jagged-40000 dataset.  The results showed a noticeable improvement from the original dataset, however only a slight improvement from the abstract-bigrams.  Specifically, the score for the support vector machine was nearly identical to the abstract-bigrams, and maximum entropy increased by ~5\% from the baseline results.

\begin{figure}[!h]
\begin{center}
\caption{tf-idf for jagged-40000 features}
\begin{tabular}{| r | c | c |}
\hline
Features & SVM & Maximum Entropy \\ \hline
32 & 54.37\% & 66.79\% \\ \hline
100 & 63.62\% & 71.84\% \\ \hline
250 & 72.1\% & 74.15\% \\ \hline
\end{tabular}
\end{center}
\end{figure}

tf-idf is a more abstract set of features tha are built off of the abstract-bigrams.  These more abstract features are more intelligent created, and the results of these features were able to further increase the accuracy of the testing algorithm for maximum entropy.  It is interesting to note, however, that the support vector machine did not gain a boost in accuracy as maximum entropy did.  Furthermore, the results gained from these features for maximum entropy were only ~2\%.  These small gains are suggestive that more intelligent features are, in fact, able to improve the results, however with more intelligent features the gains are beginning to decline.  It would still be interesting to attempt different features to determine if this trend holds, however, this set of abstract features did perform the best out of all other sets of features we had tested.

\subsection{Summary}

These results have a few interesting findings.  First, dimensionality reduction is able to converge quite early in terms of the dimensionality of the vectors after the process.  This is noted by maximum entropy converging early in the process, and the laplacian eigenmap for the support vector machine showing early convergence.  
\\A second finding is regarding the sample size used for the patents.  It is interesting to note that the jump from 3500 samples to 20000 samples noticealby imrpoved the algorithm.  However, after that point, the algorithm did not gain in its testing accuracy.  This is suggestive that even when samples are doubled, results will have a tendency to converge.  This is suprising given the fact we had initially expected the results to continue to improve has more data was included.
\\Finally, the last finding of the results are regarding the abstractness of features.  Description-Unigrams did the worst out of all possible sets of features in the construction of feature-vectors.  However, as the features tended to become more abstract in terms of one set of features being built off of an early set, the testing accuracy of the algorithms further improved.  This is noted by the move to abstract-bigrams from description-unigrams improving the testing accuracy, and furthermore the move from tf-idf being built with abstract-bigrams have a higher testing accuracy than only the abstract-bigrams. Overall, we found these finding to be interesting, and a few of them even surprising.